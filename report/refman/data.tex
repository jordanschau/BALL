\section{Data}\label{ref_data}

BALL defines five user-definable data types. These five types are {\tt
  number}, {\tt string}, {\tt team}, {\tt player}, {\tt list}. These
five data types can be stored in variables and passed to functions. In
addition to these three types, BALL also defines three more types,
{\tt bool}, {\tt null}, and {\tt stat}. Variables cannot be declared
for these three types, but the values of these types might be passed
or returned by expressions.

{\tt team} and {\tt player} objects differ from the other types
because they are the only two types whose instances contain 
{\it attributes and stats} that contain more information about the
object. Attributes return other values, themselves might be {\tt team}
or {\tt player} objects. Stats can only return {\tt number}s. Users
cannot add more stats or attributes to these two types. See tables
\ref{ref_data_player_bstats}, \ref{ref_data_battr} and
\ref{ref_data_team_bstats} for a list of builtin stats and attributes
that must be implemented.

%%% DESCRIPTION OF DATA TYPES
\begin{table}[htdp]
\begin{center}
\begin{tabular}{l p{4.5in}}
{\tt number} &
the type {\tt number} means a floating-point number. All BALL numbers
are floating point numbers. \\
\hline
{\tt string} &
A BALL {\tt string} is a sequence of characters. It can be
concatenated  but not split. \\
\hline
{\tt team} &
A {\tt team} is a collection of players. There is no way to define new {\tt
  team}s inside a BALL program. Instead all {\tt team} objects must be
loaded using the {\tt load()} builtin function. \\
\hline
{\tt player} & A {\tt player} object contains the name of the player and a
few statistics regarding the player's performance. There is no way to
define new {\tt player}s inside a BALL program. Instead all {\tt
  player} objects come from existing {\tt team} objects. \\
\hline
{\tt list} & In BALL, {\tt list} objects are not all equal in type;
List types are equal only if both lists contain the same type of
object. (hence the list type definition is recursive.) BALL lists
can only contain objects of the same type. \\
\hline
{\tt bool} & {\tt bool} values represent a truth value: one true and
one false. {\tt bool} variables cannot be declared, and the explicit
values are also hidden from the source. Instead {\tt bool} is used for
logical expressions to communicate with their parent expressions. \\
\hline
{\tt null} & Internally, the type {\tt null} is used as a wildcard
type that matches any other type regardless of what the other type
is. This type is unavailable for the user but used for builtin
functions. \\
\hline
{\tt stat} & Stats are small expressions that (internally) can be passed
around as data. Each stat is associated with either {\tt team} objects
or {\tt player} objects. \\

\end{tabular}
\end{center}
\caption{BALL data types}\label{ref_data_types}
\end{table}


%%% PLAYER STATS
\begin{table}[htdp]
\begin{center}
\begin{tabular}{|c|c|p{5cm}|}
\hline
Name & Type & Meaning\\
\hline
\texttt{IP} & Pitcher & Innings Pitched\\
\texttt{K} & Pitcher & Strikeouts\\
\texttt{H} & Pitcher & Hits Allowed\\
\texttt{BB} & Pitcher & Walks Allowed\\
\texttt{ER} & Pitcher & Earned Runs\\
\texttt{AB} & Batter & At Bats\\
\texttt{R} & Batter & Runs\\
\texttt{H} & Batter & Hits\\
\texttt{2B} & Batter & Doubles\\
\texttt{3B} & Batter & Triples\\
\texttt{HR} & Batter & Home Runs\\
\texttt{BB} & Batter & Walks\\
\hline
\end{tabular}
\end{center}
\caption{Primitive Stats for \texttt{player} objects}\label{ref_data_player_bstats}
\end{table}%


% BUILTIN ATTRIBUTES
\begin{table}[htdp]
\begin{center}
\begin{tabular}{|c|c|c|p{5cm}|}
\hline
Object Type & Name & Return Type & Meaning\\
\hline
\texttt{player} & \texttt{name} & \texttt{string} & Name loaded from CSV \\
\texttt{team} & \texttt{teamname} & \texttt{string} & Team name from
CSV \\
\texttt{team} & \texttt{BATTERS} & \texttt{list of player} & Players
marked as batters \\
\texttt{team} & \texttt{PITCHERS} & \texttt{list of player} & Players
marked as pitchers \\
\texttt{team} & \texttt{PLAYERS} & \texttt{list of player} & All
players \\
\hline
\end{tabular}
\end{center}
\caption{Builtin Attributes}\label{ref_data_battr}
\end{table}%


% BUILTIN TEAM STATS
\begin{table}[htdp]
\begin{center}
\begin{tabular}{|c|p{5cm}|}
\hline
Name & Meaning\\
\hline
\texttt{W} & Games Won\\
\texttt{L} & Games Lost\\
\hline
\end{tabular}
\end{center}
\caption{Primitive Stats for \texttt{team} objects}\label{ref_data_team_bstats}
\end{table}%

