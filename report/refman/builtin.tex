
\section{Built-in Functions}

\subsection{Load Function}
The \texttt{load} function is used to import external
comma-separated-value team files (see CSV Specification for csv format
and specification) into \texttt{team} objects in a file. The syntax
is:

\begin{verbatim}
load(string);
\end{verbatim}
Example: 
\begin{verbatim}
team Astros = load("Astros.team");
\end{verbatim}

\subsection{Rand Function}
The rand function is used to generate a random (float) number between
two given arguments. The syntax is:

\begin{verbatim}
rand(number, number);
\end{verbatim}
Example: 
\begin{verbatim}
number randomNum = rand(0,1);
\end{verbatim}

\subsection{Max Function}
The max function is used to determine the largest of two given
arguments. The syntax is:

\begin{verbatim}
max(number, number);
\end{verbatim}
Example: 
\begin{verbatim}
number larger = max(5,100); // 100
\end{verbatim}
 
\subsection{Min Function}
The min function is used to determine the smallest of two given arguments. The syntax is:

\begin{verbatim}
min(number, number);
\end{verbatim}
Example: 
\begin{verbatim}
number larger = min(5,100); // 5
\end{verbatim}

\subsection{Top Players Function}
The top function is used to generate a sublist of players representing
the top players in that list with respect to the stat. The syntax is:

\begin{verbatim}
topPlayers(number, list, player stat);
\end{verbatim}
Example: 
\begin{verbatim}
list of player top5b = topPlayers(5, Braves, AVG);
\end{verbatim}

\subsection{Top Teams Function}
The \texttt{topTeams} function is used to generate a sublist of teams
representing the top teams in that list with respect to the stat. The
list is sorted with teams having the biggest tested stat values
first. The syntax is:

\begin{verbatim}
topTeams(number, list, team stat);
\end{verbatim}
Example: 
\begin{verbatim}
list of team top5t = topTeams(5, [Braves, Dodgers, Orioles], W);
\end{verbatim}

\subsection{Bottom Functions}
The \texttt{bottom} functions are counterparts to the \texttt{top}
functions. Instead of descending sort, the resulting list is sorted
least stat value first.

\begin{verbatim}
list of player bot5a = bottomPlayers(5, Braves, AVG);
list of team bot5b = bottomTeams(5, [Braves, Dodgers, Orioles], W);
\end{verbatim}

\subsection{IsValid Function}
The \texttt{isValid} function is used to determine whether a
\texttt{from} expression returns a value successfully. When the
argument is what \texttt{from} returns in failure, \texttt{isValid}
returns \texttt{false}. Otherwise, it returns \texttt{true}.

\begin{verbatim}
isValid(any type);
\end{verbatim}
Example: 
\begin{verbatim}
isValid("a" from string[]); // false;
isValid(3 from number[]); // false;
isValid(-3 from [2, 4, 5, 1]); // false;
isValid(5 from [2, 4, 5, 1]); // true;
isValid(dodgers from [dodgers, astros, reds, yankees]); // true;
\end{verbatim}

\subsection{Sim Function}
The \texttt{sim} function is BALL's main function. When this function
is called, it takes the activated (see section
\ref{ref_stmt_activate}) simfunction (see section
\ref{ref_stmt_simdef}) and runs it on the two teams a number of
times. The team that gets returned is the team that has the larger
number of wins. The syntax is:
\begin{verbatim}
sim(team,team,number);
\end{verbatim}
Example: 
\begin{verbatim}
sim(Rockies,Giants,5); 
\end{verbatim}
